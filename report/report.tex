
\title{CS3244 Machine Learning \\ Assignment 2: Text Classification with WEKA}
\author{Shawn Tan (U096883L)}
\date{}
\documentclass[12pt]{article}
	\addtolength{\oddsidemargin}{-0.9in}
	\addtolength{\evensidemargin}{-0.9in}
	\addtolength{\textwidth}{1.7in}
	\addtolength{\topmargin}{-0.75in}
	\addtolength{\textheight}{1.1in}
\usepackage[compact]{titlesec}
\usepackage{amsmath}
\usepackage[obeyspaces]{url}
\usepackage{tikz}
\usetikzlibrary{trees}
%\titlespacing{\section}{0pt}{*0}{*0}
%\titlespacing{\subsection}{0pt}{*0}{*0}
\titlespacing{\subsubsection}{0pt}{*0}{*0}

\linespread{1.1}
\begin{document}
\maketitle
\section{Introduction}
Text classification is a common problem in the field of machine learning and Natural Language Processing (NLP). In this assignment, we were tasked to classify some posts on several newsgroups.

We were given stemmed texts from 5 newsgroups: \url{comp.graphics}, \url{comp.os.ms-windows.misc}, \url{comp.sys.ibm.pc.hardware}, \url{comp.sys.mac.hardware} and \url{comp.windows.x}. In our test set, we were given 1425 instances to classify, and 2935 training instances. Several scripts and programs were supplied to perform various tasks:
\begin{description}
	\item[fs.php and fe.php] These two PHP scripts help to extract the features from the texts using \textsc{TF-IDF} and $\chi^2$ as feature selection methods.
	\item[Formatting.exe] This program converts the \url{.txt} files created by the PHP scripts into \url{.arff} files which can be read by WEKA.
\end{description}
The end result are two \url{.arff} files that consist of features that correspond to normalised word frequencies. These are the feature vectors which the various classifiers used will be working with. In this report, we experiment with using 3 different types of classifcation algorithms: $k$-Nearest Neighbour, Naive Bayes, and SVMs. 

Our approach involves training different classifiers using each of the algorithms using the same dataset. Eventually, we take the best performing classifiers from each different algorithm, and use these classifiers together to hopefully reduce any kind of overfitting caused by any of the individual algorithms. After evaluating this classifier, we then use this to classify our test set.

We make use of version 3.7.4 of WEKA for the tasks detailed in this report.

\section{Selecting the Features}
 Setting an overly high value for feature selection may result in feature vectors that are too specific to the training set, and eventually cause overfitting. For our first experiment, we select only the top 50 keywords for each class for our feature vector. This resulted in 203 keywords in total.

Using the selected features, we extract the feature vectors from each of the newsgroup posts. Using this, we train three classifiers ($k$NN, Naive Bayes, SVM) using the default WEKA settings, and evaluate their performances before proceeding. We do this several times, with several different values of \url{fs_top_num}. Table \ref{table:fs} reports the different values we tried, and the weighted F-measure of the corresponding classifiers.
\begin{table}[h]
\linespread{1}
\label{table:fs}
\centering
\begin{tabular}{|l|c| c c c |}
\hline 
	\url{fs_top_num} & Keywords/Features   & \textbf{Naive Bayes}& \textbf{SVM} & \textbf{IBk} \\
\hline
	50	& 203	& 0.738 & 0.77 	& 0.732 \\
	100	& 428   & 0.74	& 0.801	& 0.758	\\
	150 & 641	& 0.743 & 0.814 & 0.767 \\
	200 & 857	& 0.74	& 0.822 & 0.774 \\
\hline
\end{tabular}
\caption{Experiments with the number of features used.}
\end{table}

Increasing \url{fs_top_num} by 50 at each round of testing, we performed the experiment four times. We decided to use an \url{fs_top_num} value of 200 for our classifcation task, as larger feature vectors may cause classification to take long periods of time, making repeated testing difficult. 	
	
	



\section{Tuning the Performance of Individual Classifiers}
In the following section we attempt to tune the performance of individual classifiers by adjusting the parameters for the different algorithms. For each algorithm, we evaluate the classifiers based on their performance on the training set and see how the classifiers can be improved.

For the following experiments with the classifiers, we use a 10-fold cross validation in all our evaluations.
\subsection{Naive Bayes (\url{NaiveBayes})}
The Naive Bayes classifier does not allow for much tweaking of parameters. The default (using WEKA's settings) classifier has an  F-measure of 0.74 on the training set. It is worth noting that the F-measure on class 4 was 0.809 and has a true positive rate of 0.81 for class 3 , as this may be useful in our attempt to combine the classifiers later on.

Also, observing the F-measures for the Naive Bayesian classifier over different numbers of features, we observe that the performance of the classifier does not increase much after 0.74.
\begin{table}[h]
\linespread{1}
\centering
\begin{tabular}{|l |c c c c c|}
\hline
Label 		&	TP Rate & FP Rate & Precision & Recall  & F-Measure	 \\
\hline		
\url{NB} 	& 0.739 &    0.065   &   0.75 &     0.739   &  0.74 \\
\hline
\end{tabular}
\caption{Experiments with different exponent values in \url{PolyKernel}.}
\label{table:nb}
\end{table}


\subsection{Support Vector Machines (\url{SMO})}
By default, WEKA's SMO algorithm learns a classifier with a linear decision boundary. By setting the \url{-K} parameter, we can change this to higher exponents. This is effectively mapping the original representation of data points to a different feature space. This is expected to give us better results, since many real-world problems are unlikely to be linearly separated.

The original weighted F-measure for the SVM is 0.822. Again, we see that the F-measure for class 4 is the highest, but its true positive rate for class 0 documents is at 0.86.

We ran the training set against the SMO algorithm with the \url{PolyKernel} at different exponents and evaluated the results.
\begin{table}[h]
\linespread{1}
\centering
\begin{tabular}{|l |l|c c c c c|}
\hline
Label & Exponent	&	TP Rate & FP Rate & Precision & Recall  & F-Measure	 \\
\hline
\url{SVM} 	 & 1	&		0.821 &    0.045 &     0.823   &   0.821   &   0.822  \\
\url{SVM0.5} & 0.5 &	0.748 &    0.063 &     0.75    &   0.748   &   0.748  \\   
\url{SVM1.5} & 1.5	&	0.844 &    0.039 &     0.845   &   0.844   &   0.844  \\
\url{SVM2.0} & 2.0	&	0.838 &    0.041 &     0.839   &   0.838   &   0.838  \\
\url{SVM3.0} & 3.0	&	0.831 &    0.042 &     0.833   &   0.831   &   0.832  \\
\url{SVMRBF} & $-$	&	0.751 &    0.063 &     0.787   &   0.751   &   0.749  \\

\hline
\end{tabular}
\caption{Experiments with different exponent values in \url{PolyKernel}.}
\label{table:svm}
\end{table}

Evaluating kernels of exponent 0.5, 1, 1.5, 2, and 3, we saw that the best performing SVM was when \url{-E} was set at 1.5, which performed slightly better than the default settings. This setting brings up the F-measure over all classes to over 0.8, which hopefully, gives us better performance. Beyond 1.5, the values of the weighted F-measures seem to decrease. The best reported F-measure was for class 3 at 0.869. Higher orders of exponents seem to result in poorer results, suggesting that the data does not fit well to functions of orders 2 and above. %try with 0.5

We attempt the same classification task with the RBF kernel, but obtained results poorer than the default settings for the SVM. Again, this suggests that the RBF kernel is not a good fit for the data.

There was also a \url{-L} option for the kernel in order to allow the SVM to model the function using smaller orders of the variables. We again ran the algorithm on the same set of data, except now with the \url{-L} option turned on.
\begin{table}[h]
\linespread{1}
\centering
\begin{tabular}{|l |l|c c c c c|}
\hline
Label & Exponent	&	TP Rate & FP Rate & Precision & Recall  & F-Measure	 \\
\hline
\url{SVM0.5-L} & 0.5 	& 0.828  &    0.043   &    0.831  &    0.828  &    0.829  \\ 
\url{SVM1.5-L} & 1.5	& 0.83   &    0.043   &    0.831  &    0.83   &    0.83   \\
\url{SVM2.0-L} & 2.0	& 0.836  &    0.041   &    0.836  &    0.836  &    0.836  \\
\url{SVM3.0-L} & 3.0	& 0.843  &    0.039   &    0.843  &    0.843  &    0.843  \\
\url{SVM4.0-L} & 4.0	& 0.847  &    0.038   &    0.848  &    0.847  &    0.847  \\
\hline
\end{tabular}
\caption{Experiments with different exponent values in \url{PolyKernel} using \url{-L} option.}
\label{table:svm-l}
\end{table}

Observing that there was a gradual upward trend in this case, we tried another iteration of the experiment using \url{-L -E 4.0}. This resulted in a classifier that yielded an average F-measure of 0.847

Eventually, we decided to use \url{SVM1.5} and \url{SVM4.0-L} for our SVM classifiers. Our experimental results are shown in Table \ref{table:svm}. These are listed as the weighted averages of the different shown values.
\subsection{$k$-Nearest Neighbours (\url{IBk})}
The default setting for IBk on WEKA has the $k=1$. This results in every new instance being classified the same a the first nearest neighbour it sees. We experiment with different values of $k$ to find a good classifier.

The classifier trained with the default settings gave an F-measure of 0.774, and an F-measure of 0.805 on class 0 and 0.807 on class 4. From the performance of the other classifiers on this classification, this suggests that instances in classes 0 and 4 are more easily distinguishable from the rest of the dataset.

We experiment with different values of $k$, evaluating the classifier with different $k$ values.
\begin{table}[h]
\linespread{1}
\centering
\begin{tabular}{|l |l|c c c c c|}
\hline
Label & $k$ &	TP Rate & FP Rate & Precision & Recall  & F-Measure	 \\
\hline
\url{IB1}	& 1		&0.774    &  0.056   &    0.775  &    0.774   &   0.774	  \\
\url{IB5}	& 5 	&0.762    &  0.06    &    0.77   &    0.762   &   0.763	  \\   
\url{IB10}  & 10	&0.768    &  0.058   &    0.775  &    0.768   &   0.768	  \\
\url{IB20}	& 20	&0.769    &  0.058   &    0.776  &    0.769   &   0.77 	  \\
\url{IB30}	& 30	&0.76     &  0.06    &    0.767  &    0.76    &   0.761	  \\

\hline
\end{tabular}
\caption{Experiments with the number of features used.}
\label{table:knn}
\end{table}

Using the \url{-X} function to choose $k$ did not seem to contribute much to the results of the classifier. One noticeable characteristic of this classifier was that it was much faster than the training times of the SVM. However, the results of the classifiers tend to be poorer than that of the SVM as well. 

Weighing the distances by its inverse and similarities using the \url{-I} and \url{-F} options improve the classifier's performance slightly. We turn on the options using the $k=10$ and $20$.
\begin{table}[h]
\linespread{1}
\centering
\begin{tabular}{|l |l|c c c c c|}
\hline
Label & $k$ &	TP Rate & FP Rate & Precision & Recall  & F-Measure	 \\
\hline
\url{IB10-F} & 10  &	0.783  &	   0.054  &	    0.786  &	   0.783  &	   0.783\\
\url{IB20-F} & 20 &		0.775  &	   0.056  &	    0.78   &	   0.775  &	   0.776\\
\url{IB10-I} & 10 &		0.788  &	   0.053  &	    0.791  &	   0.788  &	   0.789\\ 
\url{IB20-I} & 20 &		0.787  &	   0.053  &	    0.792  &	   0.787  &	   0.788\\ 

\hline
\end{tabular}
\caption{Experiments with the number of features used.}
\label{table:knn-l}
\end{table}

Looking at the results in Table \ref{table:knn-l}, we find that the performance of classifiers using the \url{-I} option generally do better. This may suggest that small differences in the distance between data points make a big difference for this classification problem. The fact that we see about a 2\% improvement over the previous values also suggest that there are numerous cases in which there are equal numbers of the different classes in the $k$ nearest data points. 

It should be noted that the IBk classifiers generally have nearly 0\% errors when run on their own training set. This is because the points to be classified fall directly on themselves, giving the same results. As such, testing the IBk classifier on the same set of data as the training set is unproductive.

\subsection{Conclusion}
In general, all our selected classifiers perform at over 0.75 for their F-measure, and above 75\% for their average true positive rates. In comparison, a random classifier would average at a 20\% rate for accuracy. In the case for our best performing classifier, \url{SVM4.0-L}, our accuracy is at 84\%. This is also comparable with the textbook's example from Joachims %Joachims1996
, which has an accuracy of 89\%.

\section{Combining the Classifiers}
From each type of classifier, we pick the two best performing. Since theres only one instance of the Naive Bayes classifier, we only use one. We then combine the 5 resulting classifiers into one using the \url{weka.classifiers.meta.Vote} classifier. We choose to combine them by majority vote, which means that the classification will be determined by the most common classification among the 5 classifiers.

Our chosen classifiers are:
\begin{enumerate}
	\item \url{NB}
	\item \url{SVM1.0}
	\item \url{SVM4.0-L}
	\item \url{IB10-I}
	\item \url{IB20-I}
\end{enumerate}
\begin{figure}
\centering
\usetikzlibrary{shapes,arrows}
\tikzstyle{block} = [draw, fill=blue!20, rectangle, 
    minimum height=3em, minimum width=6em]
\tikzstyle{split} = [draw, fill=blue!20, circle, node distance=2cm]
\tikzstyle{input} = [coordinate]
\tikzstyle{output} = [coordinate]
\tikzstyle{pinstyle} = [pin edge={to-,thin,black}]

% The block diagram code is probably more verbose than necessary
\begin{tikzpicture}[auto, node distance=2cm,>=latex']
    % We start by placing the blocks
    \node [input, name=input] {Instance};
    \node [split, right of=input, node distance=4cm] (split) {};
    \node [block, right of=split, node distance=4cm] (svm) {\url{SVM1.5}};
	\node [block, above of=svm] (svm-l) {\url{SVM4.0-L}};
	\node [block, above of=svm-l] (nb) {\url{NB}};
	\node [block, below of=svm] (ibk10) {\url{IB10-I}};
	\node [block, below of=ibk10] (ibk20) {\url{IB20-I}};
	\node [block, right of=svm, node distance=4cm] (maj) {\url{Vote} Majority};
	\node [output, right of=maj, node distance=4cm] (out) {};
    % We draw an edge between the controller and system block to 
    % calculate the coordinate u. We need it to place the measurement block. 

    % Once the nodes are placed, connecting them is easy. 
    \draw [draw,->] (input) -- node {instance} (split);
	\draw [->] (split) -- node {} (svm)		;
	\draw [->] (split) |- node {} (svm-l)	;
	\draw [->] (split) |- node {} (ibk10)	;
	\draw [->] (split) |- node {} (ibk20)	;
	\draw [->] (split) |- node {} (nb)		;
	\draw [->] (svm)	-- node {} (maj);
    \draw [->] (svm-l)	-| node {} (maj);
    \draw [->] (ibk10)	-| node {} (maj);
    \draw [->] (ibk20)	-| node {} (maj);
    \draw [->] (nb)		-| node {vote} (maj);
	\draw [->] (maj) -- node{classify} (out);
\end{tikzpicture}
\caption{Combining the classifiers.}
\end{figure}

In our selection of the classifiers we have to take into consideration the performance of each of the included classifier. Having multiple classifiers prone to errors would result in a correspondingly error prone combined classifier. The benefit of combining them, however, is due to the fact that the different algorithms have different strengths and weaknesses, and giving only the output which majority of the classifiers agree on is likely to improve the overall performance. The error made by one classifier could then be corrected by the other classifiers.

Combining the classifiers, we run the training set against the newly formed classifier. The obtained results are in Table \ref{table:combi}.

\begin{table}[h]
\linespread{0.75}
\centering
\begin{tabular}{|l | c c c c c|}
\hline
Class &	TP Rate & FP Rate & Precision & Recall  & F-Measure	 \\
\hline

\hline
0             &     0.998   &   0   &        0.998   &   0.998   &   0.998  \\  
1             &     0.998   &   0   &        0.998   &   0.998   &   0.998  \\ 
2             &     1       &   0   &        1       &   1       &   1      \\ 
3             &     1       &   0   &        1       &   1       &   1      \\  
4             &     1       &   0   &        1       &   1       &   1      \\  
Weighted Avg. &     0.999   &   0   &        0.999   &   0.999   &   0.999  \\ 
\hline
\end{tabular}
\caption{Class breakdown of the performance of the combined classifier.}
\label{table:combi}
\end{table}
Recall that our best individual classifier had an accuracy of 84\% while this combined classifier has an accuracy close to 100\%. This goes to show that there are certain instances for which some of the classifiers fail, and, conveniently, the majority of the other classifiers correctly classify the instance. This is in agreement with our original hypothesis that the different classifiers have different strengths in classifying the various different categories, and combining them with a majority vote gives an exceptionally accurate classification.

We must keep in mind that there is a possibility that the classifier may be over fitted for the training set. However, the chances of this might be lower, due to the fact that there is cross-validation going on in between the different classifiers, giving us a much more accurate result. As such, we attempt to use the classifier to make a prediction on the training set. 

Using the output and manually (and randomly) checking the classified instances, we found that the predicted values were fairly accurate. WEKA reported 17 instances for which classifier were not confident, and we found that the posts were generally short, or had few distinguishing features to determine its class. It was difficult to determine its category from the content, even if done by hand. 

The unsure instances in the test set can be found in Appendix \ref{unsure}.

\section{Conclusion}
Through our experiments, we see that each of the different strengths when it comes to the task of text classification. In general, SVMs do well at this task, while the Naive Bayes classifier was the worse of the three. This may have been due to the type of data presented were not discrete boolean values of whether the word was present, but frequency distributions. Analysing the data in WEKA, we see that it is difficult to have any form of classification of the data done using one attribute alone. This may have also contributed to the poor performance of the Naive Bayes classifier.

The $k$-Nearest Neighbour instance-based learning algorithm was also not very effective at predicting the classes, although using inverse distance voting, we managed to get an F-measure of around 0.79. This this suggests some of the data points are close together, and penalising points just slightly further away from the newly introduced instance improves the performance.

The SVM classifier, even with its default setting to learn a linear function to model the data, does relatively well compared to the other two algorithms. Through our experiments, we achieved the best performance when using the an exponent of 1.5, and 4.0 and lower. This exhibits some of the power of using SVMs, as simply by switching the kernel, huge gains in performance can be seen. This comes at the price of the time taken to compute the classifier. Also, our experimental data seems to suggest that the data is best approximated with a polynomial function of order 4, with perhaps larger coefficients for terms with order 1.5.

Combining the learnt classifiers, we attain a near perfect classification of the training set. This suggests that despite a maximum accuracy of 84\%, the combined power of these classifiers can correct their own errors via cross-validation amongst the classifiers. It is unfortunate that WEKA toolkit does not provide analysis tools or more detailed output for its \url{meta.Vote} classifier, as it would be interesting to look at which instances a minority of the classifiers classify wrongly, but is out-voted by the rest to give a right answer. We suspect that this occurs relatively frequently with a good spread of different classifiers making mistakes, given the huge performance gain we are seeing.

We are aware, however, that there is a danger of the individual classifiers overfitting the training set, but we believe that the exceedingly accurate performance by the meta-classifier is due to the cross-validation taking place between classifiers. Taking into account the meta-classifier's performance on the training set, together with the number of instances for which the meta-classifier is uncertain, we are confident with the predictions that have been made.

\appendix
\section{Resources}
The various scripts used to run experiments and create this report can be found here: \\
\url{https://github.com/shawntan/CS3244}
\section{Uncertain Instances}\label{unsure}
\linespread{0.25}
\subsubsection*{File Number: 3899}
\scriptsize\begin{verbatim}
from: hofkin@softwar.org (bob hofkin)
subject: re: ati build 59 driver "good"?
repli-to: hofkin@softwar.org
organ: softwar product consortium
x-newsread: tin [version 1.1 pl9]
line: 6

build 59 caus 2 except when i exit window. in fact, i have had
thi happen on all build after 44, which ship with my gatewai
system.  am i do someth wrong, or is thi problem commonli
overlook?

bob hofkin
\end{verbatim}
\subsubsection*{File Number: 3419}
\scriptsize\begin{verbatim}
from: lioness@mapl.circa.ufl.edu
subject: re: sgi sale practic (wa: crimson (wa: kubota announc?))
organ: center for instruct and research comput activ
line: 26
repli-to: lioness@ufcc.ufl.edu
nntp-post-host: mapl.circa.ufl.edu

in articl <1rr6c3$9u3@calvin.nyu.edu>, roi@mchip00.med.nyu.edu (roi smith) write:
|>	what's realli interest is that from what i can tell, the mi
|>folk in the basement with their es/9000 don't seem to be piss at ibm.
|>why?  i have no idea.  either ibm realli doe take care of their custom
|>better, or thei just have their custom brainwash better than the
|>smaller vendor do.

no, mi folk have infinit budget of death, and thei also get part
of their budget alloc "upgrad", "mainten", and "new purchas",
and a lot of ibm mainfram purchas ar actual "leas" and so
is the softwar.

basic, the engin who have tight budget, i.e. the coder and
design of a compani, bitch and moan when thei drop 15,000 on a 
sparc 1 onli to see a faster machin appear a year later.  mi type
upgrad onc everi 5-10 year, and their cost ar amort and
depreci over a longer period, and the budget offic justifi
the expens becaus thei actual us the machin for account,
payrol, etc.

now, if the budget offic wa depend on the engin for some
reason like payrol and account, you'd sure as hell see everi
engin with a new crai on hi desktop everi year. :-)

brian

\end{verbatim}
\subsubsection*{File Number: 3114}
\scriptsize\begin{verbatim}
from: bob.dohr@f174.n2240.z1.fidonet.org (bob dohr)
subject: re: good hard-disk driver for non-appl drive? (sy 7.1 compat.)
organ: fidonet node 1:2240/174 - associ mac bb, grand blanc mi
line: 33

i need to add to your messag.
i have a major problem on my hand.  i have a rodim 60+ (seri
ro3000t) extern hard drive.  rodim is out of busi, 
and not write ani more driver.  in particular, driver 
compat with system 7.1.  after talk to rodim, 
thei recommend the follow hard drive manufactur 
and their driver softwar that were compat:
 
scsi hard drive manufactur            driver softwar
----------------------------            ----------------
fwb                                     hard disk tool kit
fwb                                     hard disk tool kit - person
la cie                                  silverlin 5.2 or higher
casa blanca driver softwar             drive7
 
if anybodi ha experi with these driver softwar packag, pleas repli.
if there is sharewar out there, i would like to get my hand on it.  i would
much rather send a good develop the $25 or so, becaus most of the softwar
i mention, if purchas, would cost $125, $49, $149, and $49 respect.

thank in advanc.
bob dohr, the associ

_______________________________________________________________________________
   bring a kind word and a help spirit wherev we can, we ar...
-+- the associ - a multi-line macintosh bb in grand blanc, michigan!
   echo from fido, internet, familynet, icdmnet, k-12 - plu 2gb file
   at 313-695-6955 hst/v.32bi.
___________________________________________________________________ testifi 2.0

--  
=*=*=*=*=*=*=*=*=*=*=*=*=*=*=*=*=*=*=*=*=*=*=*=*=*=*=*=*=*=*=*=*=*=
 bob dohr - internet: bob.dohr@f174.n2240.z1.fidonet.org
\end{verbatim}
\subsubsection*{File Number: 3538}
\scriptsize\begin{verbatim}
from: vaughan@ewd.dsto.gov.au (vaughan clarkson)
subject: connect a digitis to x (repost)
organ: defenc scienc and technolog organis, salisburi, south australia
line: 32
nntp-post-host: caesar.dsto.gov.au

hi there

(i post thi to comp.window.x.intrins but got no respons, so i'm post
here.)

i'm want to connect a digitis made for pc into my workstat (an hp 720).
it is my understand the x window can understand a varieti of input devic
includ digitis tablet.  howev, thi digitis make us of the serial
port, so there would seem to be a need to have a special devic driver.

the hp manual page sai that the hp x server will accept x input from
devic list in the /usr/lib/x11/x*devic file (* = displai number).
i shouldn't think i would be abl to simpli insert /dev/rs232c as an input
devic in thi file and expect a digitis to work.  but mayb i'm wrong.  am i?

what i would like to know is: doe anybodi out there have a digitis connect
to their workstat for us as a pointer for x (rather than just as input to a
specif x applic)?  if so, what were the step requir for instal?
did you need a special devic driver?  did the manufactur suppli it?  ar
there gener public domain devic driver around?  (i understand that
digitis gener us onli a coupl of standard format.)

ani help would be greatli appreci.

cheer
- vaughan

-- 
vaughan clarkson                  ___________    email: vaughan@ewd.dsto.gov.au
engin ph.d. student              |                  phone: +61-8-259-6486
& glider pilot			       ^                    fax: +61-8-259-5254
     ---------------------------------(_)---------------------------------
\end{verbatim}
\subsubsection*{File Number: 4001}
\scriptsize\begin{verbatim}
from: ladaski@netcom.com (john j. ladaski ii)
subject: atari 1040 - sell or trade for pc
organ: netcom on-line commun servic (408 241-9760 guest)
line: 48


        i am consid sell an atari 1040 and purchas an ibm compa-
tibl.  i need to know what kind of monei or trade i can expect to get for
the atari befor i bother.  i am about to start graduat school, and that
mean i'm about to be poor!  (there's a price list for us synthes on
rec.music.maker.synth, but no equival list for comput...)

thi system is tailor-made for a midi musician.  detail follow:

  * atari 1040 st
      to 1.0
      1 mb ram
      720k floppi drive

  * supradr 20 mb extern scsi drive, 18 month old

  * 12" atari monochrom monitor

  * gener 2400 baud extern modem

  * softwar: all softwar is regist and come with manual.
      passport's master track pro, version 2.5 (sequenc softwar)
      dr. t's copyist profession (score softwar)
      first word (word processor - *not* the pd version)
      megamax's laser c, version 2.0 (program languag)
      vip profession (spreadsheet packag - low-tech lotu clone)
      partner st (desk accessori with integr calendar, cardfil, etc.)
      migraph's easi draw (an earli, pre-postscript releas)
      neodesk (improv desktop for atari st)
      univers iii (improv file selector for atari st)
      miscellan softwar (includ uniterm commun softwar)


        i will consid all price abov $900.  i am also will to
trade the atari system for a qualiti (386 or 486) pc, includ lap-top.
i own some pc hardwar, so a complet system mai not be necessari.

-- 
== john j. ladaski ii ("ii") ========================= ladaski@netcom.com ==
"great compos do not borrow -	     "talk about music is like
 thei steal."  - john ladaski	      ~ -     danc about architectur."
(quot stolen from stravinski, who    o o     - elvi costello?  lauri
 stole it from a statement made by     >        anderson?  frank zappa?
 pablo picasso about paint, who    \_/    -------------------------------
 stole it from...)			     "properti is theft." - groucho
----------------------------------------------------------------------------
"a man w/o chariti in hi heart - what ha he to do with music?" - confuciu
============================================================================
\end{verbatim}
\subsubsection*{File Number: 3880}
\scriptsize\begin{verbatim}
from: west@mail  (joe west)
subject: bb 
nntp-softwar: pc/tcp nntp
keyword: gatewai2000 
line: 4         
organ: loral data system
distribut: usa 

        i read on the bb a while back that a bb mai be start for
        gatewai2000. did a bb start, and if it did, would you let me
        know the newsgroup name. pleas send inform by e-mail.
        my e-mail address is joe_west@ld.loral.com. thank...joe west.

\end{verbatim}
\subsubsection*{File Number: 3210}
\scriptsize\begin{verbatim}
from: corbo@lclark.edu (beth corbo)
subject: re: stylewrit ii dy?
articl-i.d.: lclark.1993apr24.195357.12033
organ: lewi & clark colleg, portland or
line: 26

in articl <1993apr24.003052.6425@ultb.isc.rit.edu> bsd9554@ultb.isc.rit.edu (b.s. davidson) write:
>i bought a stylewrit ii a coupl month ago, and late, when i print
>someth, i notic white line or "gap" run through the line be
>print.  it's almost like the paper is advanc a smidg too far when
>advanc line.  
>
>i replac the ink cartridg think it might be the problem, but the line
>ar still there.  ha anyon els notic thi problem?  what's the best wai to
>get rid of it?
>

>| brian s. davidson                 | internet: bsd9554@ultb.isc.rit.edu |


  i had a similar problem with my stylewrit i (the origin!).
have you tri clean the print head? with the swii driver, it's
and option in the print dialog box. sometim i had to do it sever
time to get the crud out. ye it wast ink, but it beat those
white annoi line.
  anoth idea is to print a coupl of page with just a big
black box. it can help to get the ink flow.
  good luck!

  beth corbo

corbo@lclark.edu
\end{verbatim}
\subsubsection*{File Number: 3751}
\scriptsize\begin{verbatim}
from: mark@taylor.uucp (mark a. davi)
subject: re: look for x window on a pc
organ: lake taylor hospit comput servic
keyword: ibm pc, x window, window
line: 22

hasti@netcom.com (amancio hasti jr) write:

>>>>*but*  your perform will suck lemon run an xserver on a clone.

>i have a clone almost with no name gener 91k xstone on a 486/33mhz
>system.

show me the realist price tag...

>>>>i can get 15" tektronix xp11 termin for under $900, and the perform
>>>>is over 80000 xstone.....

>excus me, but with a 486/50 256k cach, s3 928 isa card, 8mb xs3 (x11r5) run 386bsd  you can get 100k+ xstone at 1024x768 65mhz which i doubt 

nice, but wai over $900....
my point is price/perform  not just perform...

-- 
  /--------------------------------------------------------------------------\
  | mark a. davi    | lake taylor hospit | norfolk, va (804)-461-5001x431 |
  | sy.administr|  comput servic   | mark@taylor / mark@taylor.uucp |
  \--------------------------------------------------------------------------/
\end{verbatim}
\subsubsection*{File Number: 3082}
\scriptsize\begin{verbatim}
from: whitmor@iastat.edu (kurt d whitmor)
subject: [info request] hp deskwrit & mathematica
summari: ha anyon had a problem with these?
organ: iowa state univers, am ia
line: 12

ha anyon els gotten a system error when try to print from mathematica 2.1
to the hp deskwrit. i'm us a pb170 with:	8 meg ram
						sy 7.0.1 + tuneup
						hp print driver etc....

it work find on an imagewrit i. i'd like to get as much inform as
possibl befor i send a bug report to wolfram.

thank.

-kurt (whitmor@iastat.edu)

\end{verbatim}
\subsubsection*{File Number: 3106}
\scriptsize\begin{verbatim}
from: karljo@imv.aau.dk (karl johan olsen)
subject: re: mac plu is constantli reboot!
organ: inform & mediasci, univers of aarhu, denmark
line: 22

in articl <121741@netnew.upenn.edu>, jeff@eniac.sea.upenn.edu (georg)
wrote:
> 
> :> :
> :> : basic, the mac pluse ar constantli reboot themselv, as if the
> :> : reboot button were be push.  sometim the mac is abl to fulli boot
> :
> 
> well thi thread been go long enough... i'll add a difer twist.
> 
yet anoth twist ...

i'm expirienc the same kind of problem with my se (2.5/40), although not
as frequent.

ani suggest?

------------------------------------------------------------------------
karl johan olsen                             internet: karljo@imv.aau.dk
dept. of inform and media scienc          
univers of aarhu
denmark
\end{verbatim}
\subsubsection*{File Number: 3213}
\scriptsize\begin{verbatim}
from: rt@nwu.edu (ted schreiber)
subject: rec on  mac video system -card softwar?
nntp-post-host: mac183.mech.nwu.edu
organ: mechan engin
line: 23

what would be a good platform for some fairli basic video work of the
follow natur:

read real video in for playbak in variou app's 5-10 minnut in length
basic edit featur for said video - rearang sequenc, ad grapic
slide from someth like power point etc... 

i'm not to familiar with thi stuff but would like a good system with crisp
perform.  it's for educ/promot thing so the video qualiti
should be decent.

i'm think tempest or cyclon, big drive,load o ram, floptic or 128mb
optic ?? - howev, i'm not to sure of the variou card and softwar
that out there.

pleas email ani respons,

thank

ted schreiber
mechan engin 
northwestern univers
tel: 708.491.5386 fax 708.491.3915 email: rt@nwu.edu
\end{verbatim}
\subsubsection*{File Number: 4017}
\scriptsize\begin{verbatim}
from: mostert@itu1 (9135529 mostert  a. mnr.)
subject: et4000 linear mode ??
articl-i.d.: hippo.1993apr22.201347.16763
organ: rhode univers, grahamstown, south africa
line: 10
x-newsread: tin [version 1.1 pl8]

hi 

i have heard about a linear mode for the et4000, in which the 1mb video 
memori in linearli acces instead of the usual 64k page. doe anyon
know more about thi ? how can i enabl it and to what address is the
video memori map to ?

a. mostert
stellenbosch, rsa
mostert@cs.sun.ac.za
\end{verbatim}
\subsubsection*{File Number: 3158}
\scriptsize\begin{verbatim}
from: gene@jackatak.raider.net (gene wright)
subject: outbound laptop: question look for answer
organ: jack's amaz cockroach capitalist ventur
line: 16

sinc the demis of the outbound compani, what option would exist for me 
if i were to bui on of their laptop? 

(1) sinc the outbound (2030, 2030e, etc) us mac plu rom, won't that 
severli limit us futur applic?

(2) what is a reason price for on of their laptop? the price i've 
seen seem extrem high consid the limit choic now.

(3) how reliabl have thei proven?

ani answer would be help.

--
     gene@jackatak.raider.net (gene wright)
------------jackatak.raider.net   (615) 377-5980 ------------
\end{verbatim}
\subsubsection*{File Number: 2825}
\scriptsize\begin{verbatim}
from: mike tancsa <mdtancsa@watart.uwaterloo.ca>
subject: window and ati ultra (mach8 chip)size question
content-type: text/plain; charset=us-ascii
origin: mdtancsa@watart.uwaterloo.ca
content-transfer-encod: 7bit
organ: univers of waterloo
mime-version: 1.0
x-mailer: elm [version 2.4 pl21]
content-length: 489       
line: 15



i have just upgrad from a trident tvga9000 to an ati graphic ultra (the
old mach8 chip).  i am quit pleas with the perform so far, but have
on problem.  when us window in 800x600, i have notic that the 
tile bar and scroll bar ar significantli larger than thei were when i
wa us the trident card.  is there a set in my .ini file that i can
chang to make these smaller ?  i could not find the faq for thi list...

		--mike

mdtancsa@watart.uwaterloo.ca



\end{verbatim}
\subsubsection*{File Number: 4091}
\scriptsize\begin{verbatim}
from: zander@eclips.sheridanc.on.ca (mark zander)
subject: read-onli harddriv
nntp-post-host: eclips.sheridanc.on.ca
organ: sheridan colleg, ontario, canada
line: 13

   on a few comput which we have here at sheridan colleg there ar
file which we would like to make read onli.  i have us the do attrib command
but some peopl, who carri around the attrib program in their pocket,
have still been abl to eras some of the more import file.  ar
there ani softwar packag which would make an entir drive read-onli?
an exampl, partit the drive into two partit and have the first
drive contain the import file which can be onli read and the second
drive you could both read and write.  
  ani and all enquiri or help would be appreci.

thanx.
mark.zander@sheridanc.on.ca
 
\end{verbatim}
\subsubsection*{File Number: 2933}
\scriptsize\begin{verbatim}
from: kwgeitz@s-link.escap.de (karl-w. geitz)
subject: re: data segment and memori model usag
organ: -s-link-> public mailbox, braunschweig, germani
line: 36

hello phjm, you wrote:

> firstli, doe window 3.1 in 386 enhanc mode do anyth special
> with dll that have been compil us the larg memori model?

no.

> we ar be told that even in 386 enhanc mode window
> will load dll into *real memori below 640k* and page-lock it.

no.

> my second question relat to static data insid dll. is there
> ani wai at all to get multipl instanc of the static data
> segment (dgroup?)?

no, but...

you can alloc real static data within code segment!
when you need more dynam memori you can alloc data on the global heap.

you can forget most of what wa written about memori manag. under 3.1
you have page virtual memori. you can lock everi block without hamper
the memori manag. you can us far pointer everytim without alwai lock/
unlock the memori block.

an besid: dll's ar mostli just disguis ex's, that happen to be call
by anoth task.


karl.

------------------------------------------------------------------------
karl-w.geitz, hauptstr.50, w-3320 salzgitt 1, kwgeitz@s-link.escap.de
tel: +49-5300-6701 fax: +49-5300-6499 ci: 100010,204 bix: geitzkwg
## crosspoint v2.1 ##
\end{verbatim}
\subsubsection*{File Number: 2955}
\scriptsize\begin{verbatim}
from: christian.robert@etudi.unin.ch
subject: diamond 24-window 3.1 conflict
organ: univers of neuchatel, switzerland
line: 10

hello,
i have some problem with my diamond stealth local bu graphic card.
when i try to start window my system stop and displai:
"no free file handl,cannot load command,system halt"
it's perhap a bio setup problem but i'm not us to my ami-bio setup
if somebodi can explain me;how to setup shadow video rom, other 
shadow rom ,and also how to config the two "advenc ... setup"
for a best utilis of my graphic card.
thank for ani answer.
ch.robert
\end{verbatim}



\end{document}
