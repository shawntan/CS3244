\section{Unsure Instances} > unsure_files.tex
\subsection{File Number: 3899}
\begin{verbatim}
from: hofkin@softwar.org (bob hofkin)
subject: re: ati build 59 driver "good"?
repli-to: hofkin@softwar.org
organ: softwar product consortium
x-newsread: tin [version 1.1 pl9]
line: 6

build 59 caus 2 except when i exit window. in fact, i have had
thi happen on all build after 44, which ship with my gatewai
system.  am i do someth wrong, or is thi problem commonli
overlook?

bob hofkin
\end{verbatim}
\subsection{File Number: 3419}
\begin{verbatim}
from: lioness@mapl.circa.ufl.edu
subject: re: sgi sale practic (wa: crimson (wa: kubota announc?))
organ: center for instruct and research comput activ
line: 26
repli-to: lioness@ufcc.ufl.edu
nntp-post-host: mapl.circa.ufl.edu

in articl <1rr6c3$9u3@calvin.nyu.edu>, roi@mchip00.med.nyu.edu (roi smith) write:
|>	what's realli interest is that from what i can tell, the mi
|>folk in the basement with their es/9000 don't seem to be piss at ibm.
|>why?  i have no idea.  either ibm realli doe take care of their custom
|>better, or thei just have their custom brainwash better than the
|>smaller vendor do.

no, mi folk have infinit budget of death, and thei also get part
of their budget alloc "upgrad", "mainten", and "new purchas",
and a lot of ibm mainfram purchas ar actual "leas" and so
is the softwar.

basic, the engin who have tight budget, i.e. the coder and
design of a compani, bitch and moan when thei drop 15,000 on a 
sparc 1 onli to see a faster machin appear a year later.  mi type
upgrad onc everi 5-10 year, and their cost ar amort and
depreci over a longer period, and the budget offic justifi
the expens becaus thei actual us the machin for account,
payrol, etc.

now, if the budget offic wa depend on the engin for some
reason like payrol and account, you'd sure as hell see everi
engin with a new crai on hi desktop everi year. :-)

brian

\end{verbatim}
\subsection{File Number: 3114}
\begin{verbatim}
from: bob.dohr@f174.n2240.z1.fidonet.org (bob dohr)
subject: re: good hard-disk driver for non-appl drive? (sy 7.1 compat.)
organ: fidonet node 1:2240/174 - associ mac bb, grand blanc mi
line: 33

i need to add to your messag.
i have a major problem on my hand.  i have a rodim 60+ (seri
ro3000t) extern hard drive.  rodim is out of busi, 
and not write ani more driver.  in particular, driver 
compat with system 7.1.  after talk to rodim, 
thei recommend the follow hard drive manufactur 
and their driver softwar that were compat:
 
scsi hard drive manufactur            driver softwar
----------------------------            ----------------
fwb                                     hard disk tool kit
fwb                                     hard disk tool kit - person
la cie                                  silverlin 5.2 or higher
casa blanca driver softwar             drive7
 
if anybodi ha experi with these driver softwar packag, pleas repli.
if there is sharewar out there, i would like to get my hand on it.  i would
much rather send a good develop the $25 or so, becaus most of the softwar
i mention, if purchas, would cost $125, $49, $149, and $49 respect.

thank in advanc.
bob dohr, the associ

_______________________________________________________________________________
   bring a kind word and a help spirit wherev we can, we ar...
-+- the associ - a multi-line macintosh bb in grand blanc, michigan!
   echo from fido, internet, familynet, icdmnet, k-12 - plu 2gb file
   at 313-695-6955 hst/v.32bi.
___________________________________________________________________ testifi 2.0

--  
=*=*=*=*=*=*=*=*=*=*=*=*=*=*=*=*=*=*=*=*=*=*=*=*=*=*=*=*=*=*=*=*=*=
 bob dohr - internet: bob.dohr@f174.n2240.z1.fidonet.org
\end{verbatim}
\subsection{File Number: 3538}
\begin{verbatim}
from: vaughan@ewd.dsto.gov.au (vaughan clarkson)
subject: connect a digitis to x (repost)
organ: defenc scienc and technolog organis, salisburi, south australia
line: 32
nntp-post-host: caesar.dsto.gov.au

hi there

(i post thi to comp.window.x.intrins but got no respons, so i'm post
here.)

i'm want to connect a digitis made for pc into my workstat (an hp 720).
it is my understand the x window can understand a varieti of input devic
includ digitis tablet.  howev, thi digitis make us of the serial
port, so there would seem to be a need to have a special devic driver.

the hp manual page sai that the hp x server will accept x input from
devic list in the /usr/lib/x11/x*devic file (* = displai number).
i shouldn't think i would be abl to simpli insert /dev/rs232c as an input
devic in thi file and expect a digitis to work.  but mayb i'm wrong.  am i?

what i would like to know is: doe anybodi out there have a digitis connect
to their workstat for us as a pointer for x (rather than just as input to a
specif x applic)?  if so, what were the step requir for instal?
did you need a special devic driver?  did the manufactur suppli it?  ar
there gener public domain devic driver around?  (i understand that
digitis gener us onli a coupl of standard format.)

ani help would be greatli appreci.

cheer
- vaughan

-- 
vaughan clarkson                  ___________    email: vaughan@ewd.dsto.gov.au
engin ph.d. student              |                  phone: +61-8-259-6486
& glider pilot			       ^                    fax: +61-8-259-5254
     ---------------------------------(_)---------------------------------
\end{verbatim}
\subsection{File Number: 4001}
\begin{verbatim}
from: ladaski@netcom.com (john j. ladaski ii)
subject: atari 1040 - sell or trade for pc
organ: netcom on-line commun servic (408 241-9760 guest)
line: 48


        i am consid sell an atari 1040 and purchas an ibm compa-
tibl.  i need to know what kind of monei or trade i can expect to get for
the atari befor i bother.  i am about to start graduat school, and that
mean i'm about to be poor!  (there's a price list for us synthes on
rec.music.maker.synth, but no equival list for comput...)

thi system is tailor-made for a midi musician.  detail follow:

  * atari 1040 st
      to 1.0
      1 mb ram
      720k floppi drive

  * supradr 20 mb extern scsi drive, 18 month old

  * 12" atari monochrom monitor

  * gener 2400 baud extern modem

  * softwar: all softwar is regist and come with manual.
      passport's master track pro, version 2.5 (sequenc softwar)
      dr. t's copyist profession (score softwar)
      first word (word processor - *not* the pd version)
      megamax's laser c, version 2.0 (program languag)
      vip profession (spreadsheet packag - low-tech lotu clone)
      partner st (desk accessori with integr calendar, cardfil, etc.)
      migraph's easi draw (an earli, pre-postscript releas)
      neodesk (improv desktop for atari st)
      univers iii (improv file selector for atari st)
      miscellan softwar (includ uniterm commun softwar)


        i will consid all price abov $900.  i am also will to
trade the atari system for a qualiti (386 or 486) pc, includ lap-top.
i own some pc hardwar, so a complet system mai not be necessari.

-- 
== john j. ladaski ii ("ii") ========================= ladaski@netcom.com ==
"great compos do not borrow -	     "talk about music is like
 thei steal."  - john ladaski	      ~ -     danc about architectur."
(quot stolen from stravinski, who    o o     - elvi costello?  lauri
 stole it from a statement made by     >        anderson?  frank zappa?
 pablo picasso about paint, who    \_/    -------------------------------
 stole it from...)			     "properti is theft." - groucho
----------------------------------------------------------------------------
"a man w/o chariti in hi heart - what ha he to do with music?" - confuciu
============================================================================
\end{verbatim}
\subsection{File Number: 3880}
\begin{verbatim}
from: west@mail  (joe west)
subject: bb 
nntp-softwar: pc/tcp nntp
keyword: gatewai2000 
line: 4         
organ: loral data system
distribut: usa 

        i read on the bb a while back that a bb mai be start for
        gatewai2000. did a bb start, and if it did, would you let me
        know the newsgroup name. pleas send inform by e-mail.
        my e-mail address is joe_west@ld.loral.com. thank...joe west.

\end{verbatim}
\subsection{File Number: 3210}
\begin{verbatim}
from: corbo@lclark.edu (beth corbo)
subject: re: stylewrit ii dy?
articl-i.d.: lclark.1993apr24.195357.12033
organ: lewi & clark colleg, portland or
line: 26

in articl <1993apr24.003052.6425@ultb.isc.rit.edu> bsd9554@ultb.isc.rit.edu (b.s. davidson) write:
>i bought a stylewrit ii a coupl month ago, and late, when i print
>someth, i notic white line or "gap" run through the line be
>print.  it's almost like the paper is advanc a smidg too far when
>advanc line.  
>
>i replac the ink cartridg think it might be the problem, but the line
>ar still there.  ha anyon els notic thi problem?  what's the best wai to
>get rid of it?
>

>| brian s. davidson                 | internet: bsd9554@ultb.isc.rit.edu |


  i had a similar problem with my stylewrit i (the origin!).
have you tri clean the print head? with the swii driver, it's
and option in the print dialog box. sometim i had to do it sever
time to get the crud out. ye it wast ink, but it beat those
white annoi line.
  anoth idea is to print a coupl of page with just a big
black box. it can help to get the ink flow.
  good luck!

  beth corbo

corbo@lclark.edu
\end{verbatim}
\subsection{File Number: 3751}
\begin{verbatim}
from: mark@taylor.uucp (mark a. davi)
subject: re: look for x window on a pc
organ: lake taylor hospit comput servic
keyword: ibm pc, x window, window
line: 22

hasti@netcom.com (amancio hasti jr) write:

>>>>*but*  your perform will suck lemon run an xserver on a clone.

>i have a clone almost with no name gener 91k xstone on a 486/33mhz
>system.

show me the realist price tag...

>>>>i can get 15" tektronix xp11 termin for under $900, and the perform
>>>>is over 80000 xstone.....

>excus me, but with a 486/50 256k cach, s3 928 isa card, 8mb xs3 (x11r5) run 386bsd  you can get 100k+ xstone at 1024x768 65mhz which i doubt 

nice, but wai over $900....
my point is price/perform  not just perform...

-- 
  /--------------------------------------------------------------------------\
  | mark a. davi    | lake taylor hospit | norfolk, va (804)-461-5001x431 |
  | sy.administr|  comput servic   | mark@taylor / mark@taylor.uucp |
  \--------------------------------------------------------------------------/
\end{verbatim}
\subsection{File Number: 3082}
\begin{verbatim}
from: whitmor@iastat.edu (kurt d whitmor)
subject: [info request] hp deskwrit & mathematica
summari: ha anyon had a problem with these?
organ: iowa state univers, am ia
line: 12

ha anyon els gotten a system error when try to print from mathematica 2.1
to the hp deskwrit. i'm us a pb170 with:	8 meg ram
						sy 7.0.1 + tuneup
						hp print driver etc....

it work find on an imagewrit i. i'd like to get as much inform as
possibl befor i send a bug report to wolfram.

thank.

-kurt (whitmor@iastat.edu)

\end{verbatim}
\subsection{File Number: 3106}
\begin{verbatim}
from: karljo@imv.aau.dk (karl johan olsen)
subject: re: mac plu is constantli reboot!
organ: inform & mediasci, univers of aarhu, denmark
line: 22

in articl <121741@netnew.upenn.edu>, jeff@eniac.sea.upenn.edu (georg)
wrote:
> 
> :> :
> :> : basic, the mac pluse ar constantli reboot themselv, as if the
> :> : reboot button were be push.  sometim the mac is abl to fulli boot
> :
> 
> well thi thread been go long enough... i'll add a difer twist.
> 
yet anoth twist ...

i'm expirienc the same kind of problem with my se (2.5/40), although not
as frequent.

ani suggest?

------------------------------------------------------------------------
karl johan olsen                             internet: karljo@imv.aau.dk
dept. of inform and media scienc          
univers of aarhu
denmark
\end{verbatim}
\subsection{File Number: 3213}
\begin{verbatim}
from: rt@nwu.edu (ted schreiber)
subject: rec on  mac video system -card softwar?
nntp-post-host: mac183.mech.nwu.edu
organ: mechan engin
line: 23

what would be a good platform for some fairli basic video work of the
follow natur:

read real video in for playbak in variou app's 5-10 minnut in length
basic edit featur for said video - rearang sequenc, ad grapic
slide from someth like power point etc... 

i'm not to familiar with thi stuff but would like a good system with crisp
perform.  it's for educ/promot thing so the video qualiti
should be decent.

i'm think tempest or cyclon, big drive,load o ram, floptic or 128mb
optic ?? - howev, i'm not to sure of the variou card and softwar
that out there.

pleas email ani respons,

thank

ted schreiber
mechan engin 
northwestern univers
tel: 708.491.5386 fax 708.491.3915 email: rt@nwu.edu
\end{verbatim}
\subsection{File Number: 4017}
\begin{verbatim}
from: mostert@itu1 (9135529 mostert  a. mnr.)
subject: et4000 linear mode ??
articl-i.d.: hippo.1993apr22.201347.16763
organ: rhode univers, grahamstown, south africa
line: 10
x-newsread: tin [version 1.1 pl8]

hi 

i have heard about a linear mode for the et4000, in which the 1mb video 
memori in linearli acces instead of the usual 64k page. doe anyon
know more about thi ? how can i enabl it and to what address is the
video memori map to ?

a. mostert
stellenbosch, rsa
mostert@cs.sun.ac.za
\end{verbatim}
\subsection{File Number: 3158}
\begin{verbatim}
from: gene@jackatak.raider.net (gene wright)
subject: outbound laptop: question look for answer
organ: jack's amaz cockroach capitalist ventur
line: 16

sinc the demis of the outbound compani, what option would exist for me 
if i were to bui on of their laptop? 

(1) sinc the outbound (2030, 2030e, etc) us mac plu rom, won't that 
severli limit us futur applic?

(2) what is a reason price for on of their laptop? the price i've 
seen seem extrem high consid the limit choic now.

(3) how reliabl have thei proven?

ani answer would be help.

--
     gene@jackatak.raider.net (gene wright)
------------jackatak.raider.net   (615) 377-5980 ------------
\end{verbatim}
\subsection{File Number: 2825}
\begin{verbatim}
from: mike tancsa <mdtancsa@watart.uwaterloo.ca>
subject: window and ati ultra (mach8 chip)size question
content-type: text/plain; charset=us-ascii
origin: mdtancsa@watart.uwaterloo.ca
content-transfer-encod: 7bit
organ: univers of waterloo
mime-version: 1.0
x-mailer: elm [version 2.4 pl21]
content-length: 489       
line: 15



i have just upgrad from a trident tvga9000 to an ati graphic ultra (the
old mach8 chip).  i am quit pleas with the perform so far, but have
on problem.  when us window in 800x600, i have notic that the 
tile bar and scroll bar ar significantli larger than thei were when i
wa us the trident card.  is there a set in my .ini file that i can
chang to make these smaller ?  i could not find the faq for thi list...

		--mike

mdtancsa@watart.uwaterloo.ca



\end{verbatim}
\subsection{File Number: 4091}
\begin{verbatim}
from: zander@eclips.sheridanc.on.ca (mark zander)
subject: read-onli harddriv
nntp-post-host: eclips.sheridanc.on.ca
organ: sheridan colleg, ontario, canada
line: 13

   on a few comput which we have here at sheridan colleg there ar
file which we would like to make read onli.  i have us the do attrib command
but some peopl, who carri around the attrib program in their pocket,
have still been abl to eras some of the more import file.  ar
there ani softwar packag which would make an entir drive read-onli?
an exampl, partit the drive into two partit and have the first
drive contain the import file which can be onli read and the second
drive you could both read and write.  
  ani and all enquiri or help would be appreci.

thanx.
mark.zander@sheridanc.on.ca
 
\end{verbatim}
\subsection{File Number: 2933}
\begin{verbatim}
from: kwgeitz@s-link.escap.de (karl-w. geitz)
subject: re: data segment and memori model usag
organ: -s-link-> public mailbox, braunschweig, germani
line: 36

hello phjm, you wrote:

> firstli, doe window 3.1 in 386 enhanc mode do anyth special
> with dll that have been compil us the larg memori model?

no.

> we ar be told that even in 386 enhanc mode window
> will load dll into *real memori below 640k* and page-lock it.

no.

> my second question relat to static data insid dll. is there
> ani wai at all to get multipl instanc of the static data
> segment (dgroup?)?

no, but...

you can alloc real static data within code segment!
when you need more dynam memori you can alloc data on the global heap.

you can forget most of what wa written about memori manag. under 3.1
you have page virtual memori. you can lock everi block without hamper
the memori manag. you can us far pointer everytim without alwai lock/
unlock the memori block.

an besid: dll's ar mostli just disguis ex's, that happen to be call
by anoth task.


karl.

------------------------------------------------------------------------
karl-w.geitz, hauptstr.50, w-3320 salzgitt 1, kwgeitz@s-link.escap.de
tel: +49-5300-6701 fax: +49-5300-6499 ci: 100010,204 bix: geitzkwg
## crosspoint v2.1 ##
\end{verbatim}
\subsection{File Number: 2955}
\begin{verbatim}
from: christian.robert@etudi.unin.ch
subject: diamond 24-window 3.1 conflict
organ: univers of neuchatel, switzerland
line: 10

hello,
i have some problem with my diamond stealth local bu graphic card.
when i try to start window my system stop and displai:
"no free file handl,cannot load command,system halt"
it's perhap a bio setup problem but i'm not us to my ami-bio setup
if somebodi can explain me;how to setup shadow video rom, other 
shadow rom ,and also how to config the two "advenc ... setup"
for a best utilis of my graphic card.
thank for ani answer.
ch.robert
\end{verbatim}
